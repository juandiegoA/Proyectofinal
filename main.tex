\documentclass{article}
\usepackage[utf8]{inputenc}
\usepackage[spanish]{babel}
\usepackage{listings}
\usepackage{graphicx}
\graphicspath{ {images/} }
\usepackage{cite}

\begin{document}

\begin{titlepage}
    \begin{center}
        \vspace*{1cm}
            
        \Huge
        \textbf{RoomyNite}
            
        \vspace{0.5cm}
        \LARGE
        
            
        \vspace{1.5cm}
            
        \textbf{Juan Diego Arias Toro}
            
        \vfill
            
        \vspace{0.8cm}
            
        \Large
        Despartamento de Ingeniería Electrónica y Telecomunicaciones\\
        Universidad de Antioquia\\
        Medellín\\
        Septiembre de 2021
            
    \end{center}
\end{titlepage}

\tableofcontents
\newpage
\section{Resumen}\label{intro}
Un joven caballero despierta en el nivel inferior de una torre, de estilo gótico, no sabe cómo llegó ahí, ni sabe por qué, pero está determinado a escapar, pero su única opción es subir. En el primer piso se da cuenta que la torre tiene varios pisos (cada uno es un nivel), en cada piso deberá buscar llaves, herramientas o pociones que le ayuden superar bloqueos entre las habitaciones; pero para encontrar estos recursos necesitará mover cajas, leer ciertas instrucciones que te dirán exactamente donde colocar al personaje, o hacer caminar al personaje por una ruta, tal cual como lo indican las instrucciones. De esta manera los baúles y las puertas se abrirán, y obtendrá nuevos objetos para poder abrir las puertas y avanzar al siguiente piso.

\section{Entorno}\label{intro}
Los pisos son lugares tenuemente iluminados con una antorcha. El factor de poca luz es implementado para dificultar el movimiento del personaje por el cuarto, pues podría caer en una trampa o podría pasar por alto una poción o una caja que le ayude a seguir al siguiente nivel.
\section{Amenazas}\label{intro}
Los enemigos no existen en este juego, pero los pisos cada uno cuenta con trampas que le dificultan el paso al personaje en su afán de salir de la torre.
\subsection{Cañones}
Los cañones salen de la pared y lanzan proyectiles en un intervalo de tiempo determinado (en pisos/niveles más altos el intervalo de tiempo será más reducido), después de tres golpes de cañón el juego termina
\subsection{Rios}
Los ríos se forman en el suelo de cada piso, estos se pueden encontrar en varios colores, si se usa la poción del color incorrecto en un rio (las opciones se explican mas adelante en la sección objetos) el jugador deberá comenzar desde el inicio de ese cuarto, si entra tres veces mal en el rio que no corresponde, sera el final del juego.
\section{Objetos}\label{intro}
Los objetos serán recursos o herramientas  dispuestos para superar niveles estos pueden durar para todo el juego o pueden ser únicamente para un nivel
\subsection{Palancas}
Las palancas únicamente cumplen con el objetivo de abrir puertas, cuando se baja una palanca ya es posible pasar por la puerta.
\subsection{Cajas}
Las cajas funcionan como llaves para las puertas que no se puedan abrir por medio de las palancas (leer subsección de palancas de la esta sección), su función será ser empujada por el personaje para que esta caiga en un agujeron determinado, en caso tal de encontrarse con más de una caja el color de la caja y del agujero deberá coincidir para que la puerta abra.

No se puede empujar una caja mientras se utiliza el escudo (leer subsección escudo de esta sección).
\subsection{Escudo}
El escudo permite detener los proyectiles de los cañones (leer la subsección de cañones en la sección de amenazas), y que de esta manera no ejerzan daño al personaje, el escudo será un ítem que acompaña al personaje durante todo el juego y no puede ser retirado ni destruido, este se obtiene en el piso 3 (leer sección de pisos subsección piso III).
\subsection{Pociónes}
Las pociones permiten atravesar los ríos si se recolecto la pocion correcta, estos ítems solo están disponibles hasta que el personaje pase a otro piso, si el piso requiere cruzar un río la el personaje se encontrará con una poción en ese cuarto.
\section{Pisos}\label{intro}
\subsection{Piso I}
El primer piso se compone únicamente de dos hileras de bloques sólidos, una unida a la pared de la parte superior y la otra a la pared de la parte inferior, y una palanca, para que el jugar entienda las mecánicas de movimiento del juego (ver sección de mecánicas).
\subsection{piso II}
En el segundo piso tenemos una caja (ver sección objetos subsección cajas), después de ingresar la caja a su correspondiente agujero la puerta se abrira permitiendo al personaje pasar de nivel.
\subsection{Piso III}
En el tercer piso se introducen los cañones (ver sección amenazas subsección cañones), los cuales se encuentran  dos en la pared superior y uno en la pared inferior con un intervalo de entre proyectil y proyectil de un segundo. A la entrada del piso se le entregará el personaje en escudo para que se proteja de los cañones, también en la entrada se encontrará una caja que deberá llevar hasta el final del piso para poder encontrar su agujero y abrir la puerta.
\subsection{Piso IV}
En el cuarto piso se encontrará con dos ríos el primero azul un poco más cerca de la entrada de ese piso (Un poco más a la izquierda del cuarto), después se encontrará un espacio entre río y río, después se encontrará con un río rojo. En la entrada del piso se encontrará con una poción azul, la cual le permitirá cruzar el primer río, en el espacio entre ríos se encontrará con una poción roja que le permitirá cruzar el segundo río, y llegar a una palanca y a la salida del piso.
\subsection{Piso V}
Todavia en desarrollo
\section{Mecánicas}\label{intro}
Los controles utilizados para el juego serán únicamente las teclas A, S, D, W, BARRA ESPACIADORA. La teca A movera al personaje hacia la izquierda, S lo moverá hacia la parte inferior del piso, D lo desplaza hacia la derecha, W lo moverá hacia la parte superior del piso, la Barra espaciadora le permite al personaje interactuar con los objetos (Leer sección objetos).

Las cajas serán arrastradas por el suelo, si el jugador desea mover una caja solo debe acercarse a esta y dependiendo de la dirección con la que quiera moverla hacer que su personaje la empuje manteniendo presionada la tecla de la dirección a la cual la quiere mover.

El escudo se activará con la tecla Barra Espaciadora, si no se encuentra ningún otro objeto para interactuar con el, el personaje no puede mover ninguna caja mientras tenga el escudo activado,

\end{document}
