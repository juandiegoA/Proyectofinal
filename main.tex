\documentclass{article}
\usepackage[utf8]{inputenc}
\usepackage[spanish]{babel}
\usepackage{listings}
\usepackage{graphicx}
\graphicspath{ {images/} }
\usepackage{cite}

\begin{document}

\begin{titlepage}
    \begin{center}
        \vspace*{1cm}
            
        \Huge
        \textbf{Primeros Pasos}
            
        \vspace{0.5cm}
        \LARGE
        
            
        \vspace{1.5cm}
            
        \textbf{Juan Diego Arias Toro}
            
        \vfill
            
        \vspace{0.8cm}
            
        \Large
        Despartamento de Ingeniería Electrónica y Telecomunicaciones\\
        Universidad de Antioquia\\
        Medellín\\
        Marzo de 2021
            
    \end{center}
\end{titlepage}

\tableofcontents
\newpage
\section{Ricitos de oro}\label{intro}
Después de un choque inesperado en la nave espacial una sobreviviente cae en un planeta con aire respirable y un suelo fértil que provee de mucho alimento (un Planeta ricitos de oro), logra hacer contacto con los otros miembros de la naves aliadas que llegarán en 21 días, pero el único problema de este planeta es que su atmósfera no permite la entrada directa de luz debido a que la estrella de este sistema es muy débil, lo que impide que las baterías que alimentan el localizador y el comunicador se carguen, la buena noticia es que hay otros dos planetas cercanos uno más cerca a la estrella de este sistema solar y otro más alejado, en el planeta más cercano a la estrella (Dad Bear, o DB para abreviar), en DB la atmósfera es más fina y la luz de la estrella entra más fácil, es el lugar perfecto para recargar las baterías, de manera obvia la temperatura es mucho mayor ahí, por lo tanto no crecen plantas ni es un lugar apropiado para permanecer por mucho tiempo. El tercer planeta (Mom Bear, o MB para abreviar) es un planeta más frío, pero en este mismo cayó gran parte de la nave en la que venias, por eso mismo tiene varias herramientas y recursos que serán de ayuda en caso que el refugio, el rastreador o el intercomunicador necesiten repuestos, ademas que despues de pasar mucho tiempo en DB la temperatura de la nave puede subir a niveles peligrosos por lo cual se debe pasar por MB para que la temperatura de la nave baje; las temperaturas en MB son muy bajas por lo tanto quedarse un tiempo prolongado puede llevar a hipotermia y a la muerte.

La nave en la que se llegó al planeta te permite viajar entre los otros dos planetas, pero no permite ir más lejos, además cuenta con una batería que se debe cargar en DB.

El objetivo del juego es sobrevivir los 21 días hasta que llegue la tripulación al rescate; esto se dificulta puesto que al personaje le dará hambre y se cansara, y el único lugar donde estas necesidades son saciadas es en el planeta ricitos de oro, si los niveles de cansancio llegan al límite en alguno de los otros dos planetas (DB ó MB), se acaba el juego. Hay un obstáculo más pues pese a que los planetas están cerca, hay varios asteroides alrededor, por ende de manera aleatoria durante los viaje (Que de manera normal serían automáticos), se le pedirá al jugador tomar el control; éste deberá elegir entre dos caminos y deberá escoger el que tenga el menor número de asteroides en un tiempo determinado, de escoger mal o acabarse el tiempo será el final del juego.

\section{Ruminate}\label{intro}
Un joven está en el nivel inferior de una torre, no sabe como llego ahi, ni sabe porqué, pero está determinado a escapar, pero su única opción es subir, en el primer piso se da cuenta que la torre tiene 20 pisos y que cada piso tiene 4 habitaciones, en cada habitación deberá buscar llaves, herramientas o pociones que le ayuden superar bloqueos entre las habitaciones; pero para encontrar estos recursos necesitará mover cajas, mover escaleras para llegar a lugares más altos, leer ciertas instrucciones que te dirán exactamente donde colocar al personaje, o hacer caminar al personaje por una ruta, tal cual como lo indican las instrucciones. De esta manera los baúles se abrirán, las paredes se moverán, los tapices se caerán y se descubrirán, nuevos objetos para poder abrir las puertas y avanzar al siguiente piso.
    
No todo será tan fácil pues al llegar al piso 5, el personaje comenzará a sentir que algo va detrás de él, no sabe con exactitud qué es pero sabe que es malo y peligroso; si pasa mucho tiempo en un piso esta maldad lo alcanzara en cuya casa será el fin del juego. También se debe tener en cuenta que el personaje cuenta con 3 corazones, cada corazón está dividido en 4. Mover cajas consume corazones, hay dos tipos de cajas, piedra y madera; mover una caja de madera consume un cuarto de un corazón, y mover una caja de piedra medio corazón. Se puede reponer corazones bebiendo pociones, o reiniciando el piso (lo que restaura todos los corazones).

Despues del piso 10 nace un nuevo problema, pues la torre cuenta con defensas que atacan al personaje mientras hace su búsqueda de los  recursos y los materiales; hay dos tipos de defensas unas toman tiempo en volver a recargar y otros hacen ataques constantemente; pero también se le entregará un escudo que le permite defenderse; pero mientras se esté usando el escudo no se puede mover cajas, ni escaleras, ni usar pociones; las defensas de una habitación se pueden desactivar, aunque esto requiere tiempo, y el ser que está detrás del personaje no pierde el tiempo.

El objetivo del juego es llegar al piso 20, donde las verdades serán reveladas.

\section{The Near}\label{intro}
Este es un juego de plataformas, con los manejos y las pruebas típicas de estos mismos, el juego sería para dos personas, cada jugador tiene 30 segundos, para hacer avanzar al personaje lo más que pueda 

Ganará el que logre llevar al personaje hasta la meta que tiene trazada, si llega a caer en algunas de las trampas, se le penalizará volviendo una parte del camino mientras el tiempo sigue corriendo. Conforme avance entre las plataformas estas se harán más difíciles de superar.

El primer jugador, en su primer turno dispondrá de un minuto, pues esto ayuda a que los jugadores se encuentren en condiciones más parejas,

\end{document}
